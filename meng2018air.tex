\documentclass{article}
\usepackage[utf8]{inputenc}

\title{Spatiotemporal forecast of NO2 in Madrid, Spain}
\author{Meng Lu}
\date{October 2017}

\usepackage{natbib}
\usepackage{graphicx}
\usepackage{cleveref}
\begin{document}

\maketitle


\section{Introduction}

The number of air quality measurement stations is often limited to quantify air pollution continuously in space and time. Other data sources such as physical and chemical model simulations and spatiotemporal correlations may play an important role in improving air pollution interpolation. In this study, we investigated different Spatiotemporal and non-spatiotemporal statistical models and integrated historical records and predictions from a MACC physi-chemical model simulations to station measurements to predict air quality pollution.

We compared the results of air prediction from linear and non-linear regression models, geostatistical models, and random forest models. The linear and non-linear regression models assume no spatial correlation and purely predict using the MACC model simulations. We consider an ordinary kriging and an universal Kriging model to integrate spatial correlation and predict air pollution. For each linear/nonlinear regression, geostatistical, and random forest models, we attempted different methods and attempted to achieve the best results from these models.    

The objectives are to find the best prediction method with available data. The objective is guided by three research questions:

\begin{itemize}
\item Can spatial correlation between a limited number of air quality measurement stations be used to improve air quality prediction? 

\item Can the MACC model simulation improve air quality prediction?

\item How can we construct the random forest model to predict air quality with station measurements, MACC simulation, and temporal variables? Does the inclusion of time series characters, such as trend and harmonic terms improve model prediction? 

\end{itemize}


 
 
\section{Data}

Three years of hourly NO2 station measurements and MACC simulations are available. There are in total 24 stations distributed in Madrid, Spain. These station measurements are used to to train models and to validate the modeling results.

The distribution of MACC forecast of NO2 is very close to Gaussian. The distribution of the measured NO2 is a skewed Gaussian distribution with mean around 33.


\section{Method}
We consider a variety of linear, non-linear regression models, ordinary and universal kriging methods, and random forest methods to predict NO2 spatiotemporally. Each of these methods are described in detail.  

The NO2 predicting methods considered and compared are:
\begin{enumerate}
    \item Ordinary kriging of MACC simulation
    \item Linear regression model with MACC, harmonic terms and time of day (TOD) as potential independent variables.
    \item Polynomial and GAM (general additive model) with MACC as independent variables. 
    \item Universal Kriging with MACC simulation and potentially coordinates as variable.
    \item Random forest prediction with MACC and different temporal variables. 
     
\end{enumerate}


\subsection{Interpolation of MACC simulation} 
The MACC simulations show strong spatial correlation. In order to assimilate more information from historical MACC simulations, We computed a pooled variogram from MACC forecasts of 200 sampled time stamps. The variogram is automatically fitted with the initial parameters of nugget, sill and range set as \citet{automap}. An Ste (Matern, M. Stein’s parameterization
) model is used to interpolate MACC forecasts to the city grid and to each of the measuring stations. 

\subsection{Linear regression model with MACC as an independent variable.}

The models in this section investigate linear or non-linear relationship between MACC simulation and station measurements. Linear, polynomial, and general additive models are attempted. (Adjusted) R square and model complexity were considered in model selection.  
  
Five models are attempted to investigate the relationship between station measurements and spatially interpolated MACC forecast. Model 2a and 2b are linear regression models. In model 2b, TOD (time of day) and harmonic terms are  used as additional independent variables. The models are built with station measurements and corresponding MACC simulations from 2012-01-01 to 2015-12-31 (table 1).   

The harmonic terms   
$|sin(2 \pi wt) + \phi|$, 
with $w = 1 / (356 * 2 * 2)$ are used to fit the half year positive cyclical pattern. 45\% percent of variance could be explained by this model ($lm (station measurements \sim MACC + |sin(a)| + |cos(a)|)$ ).  

\begin{table}[h!]
\centering
\begin{tabular}{ c c c }
Model & Method & Independent variable\\ \hline 

2a &linear regression & MACC   \\
2b &linear regression & MACC, TOD, Harmonics   \\
2c1 & second order polynomial & MACC \\  
2c2 & third order polynomial & MACC   \\ 
2d & general additive model & MACC  \\  \hline
\end{tabular}
\caption{ Linear and non-linear regression models and the independent variables of the model. TOD: time of day. Harmonics: $ | sin(a)| + |cos(a)|$ } 
\label{table:1}
\end{table}

\begin{figure}[!h]
\includegraphics[scale = 0.5]{diaM2.png}
\caption{The workflow of model 2. The model 2a, which is a linear regression model with MACC as independent variable has recieved the best results and used to compare with models using other methods. }
\label{fig:LR}
\end{figure}



\subsection{Universal Kriging}
The basic downscaling models do not utilise spatial information. We attempted with
Universal Kriging to utilise spatial correlations.
Two models are attempted: 1) with residual variograms with MACC simulations as a regressor, and 2) with residual variograms with MACC simulations and locations as regressors.

We sampled 300 time stamps, compute a pooled variogram (average over the semivariance of each time
stamp).

\subsubsection{Manually fitting a variogram model to the variogram}
As no spatial correlation is revealed in the variogram, fitting a variogram model automatically to it is not
useful. We define the sill and nugget of a variogram model to fit to the variogram, then apply universal
kriging using this variogram model to interpolate station measurements to the city grid and forecast to the
next halfday.

\subsubsection{Universal kriging models}

\begin{figure}
\includegraphics[width=\columnwidth]{diaM3.png}
\caption{workflow the universal krigging models. Model 3a: using station measurements and MACC simulations at the same time to fit a model, then using MACC simulations at time t+1 to predict. Model 3b: Using station measurements and MACC simulations at the same time, as well as locaitons to fit a model, then using MACC simulations at time t+1 as well as locations to predict. Model 3c: Using all the measurements (about 4 years of halfday data) to fit a linear regression model, then apply universal Kriging with fixed coefficients from the linear regression model. The model is applied to the MACC simulations at time t+1 to predict air quality.}
\label{fig:UK}
\end{figure}

\cref{fig:UK} shows the workflow the model 3a-c. The first two models (3a and 3b) use spatial data from the most recent time stamp to fit the data. The drawback of these two models is that the measurements or MACC forecasts at one or more stations may strongly affect the linear regression model (i.e. the relationship between MACC forecasts and station measurements may be inverted). These models 3a and 3b are more suitable to the situation that more stations (e.g. more than 100) are available or when there is a lack of historical time series data. In this study, we used data from all the locations and times to fit the linear regression model, and applied Kriging to the linear regression model residuals.


 
\subsubsection{Residual variogram} 
The residual variogram of a linear regression model with MACC simulations as a regressor.
It could be observed that the spatial correlation is not revealed from the 24 station measurements. Three
possible reasons are: 1) there are too few observations (stations), 2) there is a lack of information from
short-distance placed stations, 3) due to the design of the air pollution measuring station network, the stations are placed where the pollution is suspected to be high, e.g., near factories, in a canyon.
distance semivariance
 
\subsubsection{pooled variogram}
Pooled variogram is computed from spatiotemporal points, which uses more information than only using data
from a time stamp and reduces the influence of extreme values.
Firstly we sample 300 time stamps that contain no missing data. Then, we compute a pooled residual variogram (average semivariance) of (1) a linear regression model with MACC simulations as a regressor, and (2) a linear regression model with macc simulations as a regressor.

The pooled residual variogram with MACC as independent variables (figure 5) contains a little more
information than the variogram of a time stamp. Spatial correlation is not shown in the pooled variogram.
The pooled residual variogram (figure 6) with MACC and locations as independent variables also does not
reveal spatial correlation.

\subsection{Random forest}

At last, we used a machine learning method to find a relationship between NO2 station measurements, MACC simulation, and other temporal variables. Specipically, we used MACC simulation,  day of year (DOY), time of day (TOD), lagged MACC simulations and harmonic terms as independent variables in random forest models.  

The R package "ranger" is used to perform random forest. In this study, it is found that ranger is faster than the "randomforest package". 

 Table 1 shows the variables that are used in each model. The harmonic term that is used in model 2  ($|sin(2 \pi wt) + \phi|$) is used in the one of the random forest model. The differences between model 4.2a and model 4.2b is that the model 4.2b treats DOY as factors while model 4.2a treats it as numbers. As using a lagged MACC (as is used for Ozone) to forecast Ozone obtains the lowest accuracy, and is slow, for NO2 this model is taken off.  



\begin{table}[h!]
\centering
\begin{tabular}{ c c  }
Model &  Independent variable\\ \hline 

 4.1 & MACC simulations, DOY, harmonic term, TOD \\
 4.2a & MACC simulations, DOY, TOD (factors) \\
 4.2b & MACC simulations, DOY (factors), TOD (factors) \\ 
 4.3 & DOY, MACC simulations \\
 4.4 & MACC simulations  \\  \hline
\end{tabular}
\caption{Independent variables used in each model. DOY: day of year, TOD: time of day  } 
\label{table:rf}
\end{table}

 

\section{Accuracy assessment}

The first half of the data (i.e. the first 1.5 years of time series, in total 1420 observations) are used to train the forest. The second half of the data (i.e. the last 1.5 years of time series, in total 1428 observations) are used to test the model. The missing values are filled using spline interpolation.  
The error of Model 3c is accessed different from other models, as spatial errors are assessed. It is not feasible to compare station measurements and predictions at each station because the prediction at each station are very close to the station measurements (i.e., the prediction errors will be very small at the stations).  


\section{Results}

Errors (differences between predictions and station measurements) of applying the developed models (the best model of each report) to the testing set (the second half of the all the time series). Random forest models obtained lower errors in terms of mean and median. Model 4.3 obtained the best accuracy, and it is simple.  
\begin{table}[h!]
\centering
\begin{tabular}{ c l l c c c c c}
 Model &Method& independent variables& Meidan & IQR& Mean&  RMSE& MAE \\
 \hline
 1     &OK & \-                 & -18.29 &	22.74&	-22.20 &	28.76 &	21.13\\
 2a    &LM & MACC                  & 4.3   & 23.2     & 5.32 & 19.73 & 16.00 \\
 2b    &LM & MACC, har, TOD   & -2.51 & 26.68 & -5.90 & 23.09 & 17.44\\ 
 3c    &UK & MACC             & -6.26 & 35.2 & -0.75 & 31.10 & 24.28\\
 4.1   &RF & DOY, MACC, TOD, har &  6.14 &	20.84 &	6.65 &	20.34 &	15.39\\
 4.2a  &RF & DOY, MACC, TOD                     & 6.60 &	20.89	& 7.06 &	20.07 &	15.16 \\
 4.2b  &RF & DOY(factor) MACC, TOD  & 6.61&	20.83	& 7.07 &	20.08	& 15.16 \\
 4.3   &RF & DOY, MACC              & 6.03&	21.64	& 6.50 &	20.58	& 15.76 \\
 4.4   &RF & MACC                   &6.01 &	23.63	& 6.56 &	22.21	& 16.91 \\
\hline
\end{tabular}
\caption{Accuracy assessment results of the best models of method OK (ordinary kriging), LM (linear regression) and UK (universal kriging), and all the models of the RF (random forest) regression. MACC: MACC simulations. TOD: time of day. DOY: day of year. har: harmonic terms.  
 } 
\label{table:result}
\end{table} 
 
 
 
\subsection{MACC interpolation} 
Interpolation of MACC forecasts: Strong spatial correlation are shown between MACC forecasts for all the parameters. Relatively low interpolation error.

\subsection{Linear and nonlinear regression model}  
The linear relationship between MACC forecasts and station measurements of NO2 is not so strong. Polynomial model of different orders and a general additive model were attempted. Harmonic terms were also used to fit the model.  

\subsection{Universal Kriging model}  
There are 24 stations available for NO2 and we calculated pooled residual variograms to draw more information from historical measurements; however, the sample size is still small for identifying spatial correlations. Eventually, we manually assigned the sill and nuggest of a variogram model by observing the pooled variogram, assuming spatial correlation exists within shorter distances. There are spatial trends for all the variables, therefore locations are used as regressors together witht the MACC simulation.  

\subsection{Random forest model}  
Random forest regression: One or more of the variables of MACC simulation, day of year (DOY), time of day, lagged MACC simulations and harmonic terms are used as independent variables. Random forest methods have obtained similar results with different parameters. The model 4.3, which uses MACC and DOY could be the most favorable random forest model due to its simplicity and high accuracy 


\section{Discussion}
It might be of interest to compare different randomforest implementaions. 

\section{Conclusion}

In summary, the linear regression model 2a and the random forest 4.3 obtain the lowest RMSE (the best accuracy and lowest variance). Spatial correlation may be used in random forest (as distance) to further improve the accuracy. The linear regression models and random forest models significantly improve from model 1, indicating the integration of station measurements can improve model accuracy. 

The error of Model 3c is accessed different from other models, as spatial errors are assessed. It is not feasible to compare station measurements and predictions at each station because the prediction at each station are very close to the station measurements (i.e., the prediction errors will be very small at the stations). It is possible that model 3c has some advantage when applying the model to predict the air quality of the whole city grid.    

\bibliographystyle{plain}
\bibliography{references}
\end{document}
